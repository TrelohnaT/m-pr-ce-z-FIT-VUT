\documentclass[a4paper,11pt,twocolumn]{article}
\usepackage[utf8]{inputenc}
\usepackage{hyperref}
\usepackage[czech]{babel}
\usepackage[left=1.5cm,text={18cm, 25cm},top=2.5cm]{geometry}
\usepackage[IL2]{fontenc}
\usepackage{times}
\usepackage{amsthm}
\usepackage{amsmath}
\usepackage{amssymb}


\begin{document}
\begin{titlepage}
\begin{center}
    \huge
    \textsc{Fakulta informačních technologií}\\
    \textsc{Vysoké učení technické v Brně}\\
    \vspace{\stretch{0.382}}
    Typografie a publikování-2.projekt\\
    Sazba dokumentů a matematických výrazů
    \vspace{\stretch{0.618}}
\end{center}
{\large 2019 \hfill Václav Doleček(xdolec03)}

\end{titlepage}


\section*{Úvod}
V této úloze si vyzkoušíme sazbu titulní strany, matematicých vzorců, prostředí a dalších textových struktur obvyklých pro technicky zaměřené texty (například rovnice (1)
nebo Definice 1 na straně 1). Pro odkazovaní na vzorce
a struktury zásadně používáme příkaz \verb|\label| a \verb|\ref| případně \verb|\pageref| pokud se chcmeme odkázat na stranu výskytu.

Na titulní straně je využito sázení nadpisu podle optického středu s využitím zlatého řezu. Tento postup byl
probírán na přednášce. Dále je použito odřádkování se
zadanou relativní velikostí 0.4 em a 0.3 em.

\section{Matematický text}

Nejprve se podíváme na sázení matematických symbolů
a výrazů v plynulém textu včetně sazby definic a vět s využitím balíku amsthm. Rovněž použijeme poznámku pod
čarou s použitím příkazu \verb|\footnote|. Někdy je vhodné
použít konstrukci \verb|\mbox{}|, která říká, že text nemá být
zalomen.

\paragraph{Definice 1.}
Zásobníkový automat \emph{(ZA) je definovaný jako sedmice tvaru} $A=$($Q$, $\Sigma$, $\Gamma$, $\delta$, $q_0$, $Z_0$, $F$), kde:

\begin{itemize}

\item $Q$ \emph{je konečná množina} všech vnitřních (řídících) stavů

\item $\Sigma$ \emph{je konečná} vstupní abeceda,

\item $\Gamma$ \emph{je knonečná} zásobníková abeceda

\item $\delta$ \emph{je} přechodová funkce $Q \times(\Sigma \cup \{ \epsilon \})\times \Gamma \to 2^{Q\times\Gamma ^*}$

\item $q_0\in Q$ $je$ počáteční stav, $Z_0 \in \Gamma je$ startovní symbol \linebreak zásobníku $a F \subseteq \Gamma je$ $mnozina$ koncových stavů.

\end{itemize}

Nechť $P = $($Q$, $\Sigma$, $\Gamma$, $\delta$, $q_0$, $Z_0$, $F$) je zásobníkový automat. \emph{Konfigurací} nazveme trojici $(q, \omega , \delta ) \in Q \times \Sigma^* \times \Gamma^*$, kde $q$ je aktuální stav vnitřního řízení, $\omega$ je dosud nezpracovaná část vstupního řetězce a $\alpha = Z_{i1}, Z_{i2} \dots Z_{ik}$ je obsah zásobníku\footnote{$Z_{i1}$ je vrchol zásobníku}.

\subsection{Podsekce obsahující větu a odkaz}

\paragraph{Definice 2.}
Řeřězec $\omega$ nad abecedou $\Sigma$ je přijat ZA \emph{A jestliže}$(q_0, \omega, Z_0) \vdash_A^* (q_0, \epsilon, \delta)$ \emph{pro nějaké} $\gamma \in \Gamma^*$ \emph{a} $q_F \in F$. \emph{Množinu L}$(A) = \{ \omega$ $|$ $\omega$ \emph{je přijata ZA A}$\} \subseteq \Gamma^*$ \emph{nazýváme} jazyk přijímaný TS \emph{M}.
\\

Nyní si vyzkoušíme sazbu vět a důkazů opět s použitím balíku \verb|amsthm|.

\paragraph{Věta 1.}
\emph{Třída jazyků, které jsou přijímány ZA, odpovídá} bezkontextovým jazykům.

\begin{proof}
V Důkaze vyjdeme z Definice 1 a 2.
\end{proof}

\section{Rovnice a odkazy}

Složitější matematické fomulace sázíme mimno plynulý text. Lze umístnit několik výrazů na jeden řádek, ale pak je třeba tyto vhodně oddělit, například příkaze \verb|\quad|.
\\

$\sqrt[i]{x^3_i}$ kde $x_i$ je \emph{i} -té sudé číslo splňující $x_i^{2-x_i^{i^2}} \le x_i^{y_i^3}$
\\

V rovnici (1) jsou využity tři typy závorek s různou explicitně definovanou velikostí.

\begin{eqnarray}
    x & = & \left[ \Big\{ \big[ a + b \big] * c \Big\}^d \ominus 1 \right]^{1/2}
    \\
    y & = & \lim_{x\to \infty} \frac{\frac{1}{log_10 x}}{sin^2x + cos^2x} \nonumber
\end{eqnarray}

V této větě vidíme, jak vypadá implicitní vysázení li-\linebreak
mity $lim_{n \to \infty}$ v normálním odstavci textu. Podobně\linebreak
je to i s dalšími symboly jako $\prod_{i=i}^{n} 2^{i}$ či $\bigcap_{A \in B} A.$ V pří-\linebreak 
padě vzorců $\lim \limits_{n\to \infty} f(n)$ a $\prod\limits_{i=1}^{n} 2^{i}$
jsme si vynutili \linebreak méně úspornou sazbu příkazem \verb!\limits!.


\begin{eqnarray}{cc}
    \int_b^a g(x) dx & = & -\int \limits_a^b \mathit{f}(x) dx
    \\
    \overline{\overline{A \land B}} & \Leftrightarrow & \overline{\overline{A} \lor \overline{B}}
\end{eqnarray}


\section{Matice}

Pro sázení matic se velmi často používá prostředí \verb|array|
a závorky(\verb|\left.\right|).
$$
\left[
\begin{array}{ccc}
 & \beta + \gamma & \hat{\pi}\\
 \vec{a} & \overleftrightarrow{AC}
\end{array}
\right] = 1 \Longleftrightarrow \mathbb {Q} = \mathbf {R}
\\
$$
\[
\mathbf{A} = \left|
\begin{array}{cccc}
     a_{11} & a_{12} & \dots & a_{1n}  \\
     a_{21} & a_{22} & \dots & a_{2n}  \\
     \vdots & \vdots & \ddots & \vdots \\
     a_{m1} & a_{m2} & \dots & a_{mn}
\end{array}
\right| = \begin{array}{cc}
    \emph{t} & \emph{u} \\
     \emph{v} & \emph{w}
\end{array} = \emph{tw} - \emph{uv}
\]
Prostředí \verb!array!lze úspěšně využít i jinde.

\[
\binom {n}{k} = \left\{
\begin{array}{ll}
	0 & $pro $ k~< 0 $ nebo $ k~> n \\
	\frac{n!}{k!(n-k)!} & $pro $ 0 \leq k~\leq n \\
\end{array} \right. 
\]






\end{document}
